\Conclusion 
\label{ch:conclusion}

% MF CORRECTION DONE

Le travail dont ce mémoire fait l'objet est une étude exploratoire du cadre de travail local multirésolution mis au point en utilisant la transformée de Riesz. D'habitude utilisés dans le domaine du traitement du signal, ces outils sont ici examinés pour faire l'analyse de textures présentant des structures irrégulières, textures qui sont encore aujourd'hui en enjeu non résolu du domaine de la synthèse de texture procédurale~\cite{guehl_semi-procedural_2020}. Le but de cette étude est de voir si de tels outils peuvent aider à résoudre, au moins partiellement, le problème de la synthèse de texture procédurale de textures à structure irrégulière.

\bigskip

Une mise en contexte et exposition au domaine et à la problématique de la synthèse de texture à structure irrégulière est faite en introduction~\ref{ch:introduction}. Quelques approches existantes y sont présentées et des notions nécessaires à la compréhension du problème sont abordées. Un aperçu plus approfondi des méthodes employées dans ce travail est donné dans un second temps au chapitre~\ref{ch:chapitre1}. La théorie du signal monogène utilisant la transformée de Riesz est présentée, en mettant l'accent sur le concept d'information locale à extraire d'une image et la signification qui peut lui être attribuée. La description de l'outil d'analyse local multirésolution des pyramides d'images de Riesz est ensuite faite, en détaillant les différentes étapes de construction et choix d'implémentation. Enfin la fonction de congruence de phases issue du modèle d'énergie locale de Morrone \textit{et al.}~\cite{morrone_feature_1987, morrone_mach_1986} et son importance pour l'analyse de la structure d'une image sont discutées. Une méthode de calcul adaptée au cadre de travail choisi est aussi proposée. Dans un dernier temps, les différentes applications de ces outils sont présentées au chapitre~\ref{ch:chapitre2}. La décomposition en pyramide d'images permet la mise en place d'une méthode de sélection fréquentielle du contenu d'une image, utilisée pour faire du gommage de contenu indésirable dans des textures. Un échantillonneur préférentiel préservant la congruence de phases, inspiré de Pharr \textit{et al.}~\cite{pharr_physically_2023}, est ensuite dérivé et mis en application dans une méthode de synthèse de texture, pour tenter de synthétiser du contenu à structure irrégulière. Les résultats obtenus et les limites de notre approche sont discutées à la suite de cela.

\bigskip

Finalement, quelques perspectives de travaux futurs et pistes d'améliorations, relevées lors de cette étude, sont ici proposées. Une continuation de ce travail serait de s'intéresser à l'orientation locale qui découle du signal monogène. Cette grandeur a en effet été mise de côté dans cette étude pour se concentrer sur la phase, mais elle pourrait être utilisée comme information supplémentaire pour guider la synthèse de texture. Il a été discuté que la congruence de phases seule n'était pas suffisante pour bien préserver les éléments de structure d'une texture~\ref{par:discussion-synthesis}, l'utilisation de l'orientation est une piste à explorer pour complémenter la congruence. Reinhardt \textit{et al.}~\cite{reinhardt_multi-scale_2022} ont d'ailleurs montré que l'orientation extraite du signal monogène à l'aide de la transformée de Riesz est un bon outil pour certaines tâches de traitement d'image comme la détection d'orientation d'éléments ou la détection de défauts dans un tissu. Pour continuer dans la direction d'une synthèse capable de traiter des textures à structure irrégulière, travailler à mieux préserver des corrélations locales entre les pixels des textures serait pertinent. Plusieurs méthodes de la littérature s'intéressent à ce problème~\cite{cavalier_local_2019, heitz_high-performance_2018}, intégrer l'échantillonneur mis au point dans ce travail à ces travaux pourrait donner lieu à une méthode moins expérimentale et plus employable en pratique. Enfin toujours dans cette optique, la recherche de techniques de disposition de contenu pour la création de nouvelle structure lors de la synthèse reste un sujet ouvert. Sa résolution serait une grande avancée dans la synthèse de texture à structure irrégulière et donnerait lieu à des méthodes que notre échantillonneur pourrait bien complémenter.