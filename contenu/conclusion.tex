% DEBUT DE LA CONCLUSION
\Conclusion \label{chap:conclusion}



\section{Discussion et perspectives}

- résolution des problèmes de filtrage lors de la reconstruction en temps réel dans les shader (changement de méthode de stockage ? faire du filtrage sur les atlas de textures ?)
- meilleure structure pour l'échantillonneur PC preserving (periodic tiling c'est lame)
- meilleure quantification de la PC ? Ici, on ne fait que séparer en intervalle de taille constante, p-e devrions nous utiliser les quantiles et la fonction de répartition à la place ?
- informations de riesz : orientation serait à utiliser (voir papier sur l'estimation de l'orientation en multi-didmension avec Riesz)
- meilleure approximation de Riesz (travaux LÉO)
- automatiser et améliorer la sélection de structure par sélection de niveaux de la pyramide


\section{Conclusion}
% FIN DE LA CONCLUSION