% -------------------------------------------------
% PAGE DE TITRE
% -------------------------------------------------

%
% ATTENTION : ne pas utiliser les macros \title \author ou \date !!!
%
\auteur{Vivien Gagliano}
\titre{Étude locale multirésolution d'images à structure irrégulière pour la synthèse de texture procédurale}

% n'existe que si la thèse est rédigée en anglais (pas obligatoire)
\englishTitle{Title in english}


% n'existent que si l'option cotutelle est présente
\institutionPartenaire{Université XYZ} 
\departementPartenaire{au département (centre, institut) ABC} 
\gradePartenaire{docteur en informatique} 

%\dedicace{À quelqu'un de significant} %%% ajouter un item de dédicace

% -------------------------------------------------

% DESCRIPTION DU SOMMAIRE (EN FRANCAIS) -----------
\sommaire{
En informatique graphique, des méthodes de création de contenu automatisées sont nécessaires pour faire face au besoin grandissant de contenu pour les environnements virtuels. Les détails des objets des scènes virtuelles sont classiquement ajoutés à l'aide de textures. La synthèse de texture en permet la génération automatique, en utilisant de l'aléatoire pour apporter de la variété dans le contenu créé. Les textures présentant de la structure irrégulière comme les sols pavés sont un enjeu non-résolu de la synthèse car la reproduction fidèle de la structure est une problématique difficile.

\bigskip

Le travail décrit dans ce manuscript vise à l'amélioration de la synthèse de texture à structure irrégulière en utilisant des outils du domaine du traitement d'image. Le cadre de l'analyse locale multirésolution des pyramides de Riesz est utilisé pour extraire de l'information sur la structure et mieux reproduire l'apparence des textures. Une mesure mise au point pour la détection de bords, la congruence de phases, est employée pour formuler une méthode de synthèse préservant des corrélations désirables de l'exemple d'entrée. Plusieurs applications de ces outils sont présentées et discutées.
}
\motsCles{informatique graphique; synthèse de texture; analyse multirésolution; transformée de Riesz; congruence de phases; détection de bords}

%  optionnel, valide si votre document est en anglais 
%  optionnal, valid only if your document is written in english, you have to provide a french abstract 
%  option	english du document
\abstract{Write here your abstract in english}
\keywords{Put your keywords here separated by commas}


% REMERCIEMENTS -----------------------------------
\remerciements{
Le manuscript présenté ici est le fruit d'un travail qui n'est pas que le mien. J'aimerais remercier les personnes qui se sont penchées sur ce travail et qui ont pris de leur temps pour écouter, lire, réfléchir et discuter à la synthèse de texture avec moi.

\bigskip

Guillaume, pour avoir suivi mes travaux depuis le début. Nicolas, pour sa pédagogie et son enseignement des statistiques. Mes relecteurs et relectrices, grâce à qui ce document est bien meilleur que ce qu'il était initialement.

\bigskip

Un grand merci aussi à toutes les machines à café qui m'ont alimenté. Merci à CRISCO, dictionnaire des synonymes en ligne qui a Ô combien étayé le vocabulaire de mon texte. Merci à Textures.com, base de données gratuite de textures et matériaux qui a embelli mes exemples (lorsque non précisé, les textures utilisées dans ce manuscript sont la courtoisie de Textures.com).
}

% LISTE DES ABREVIATIONS --------------------------
%% Insérez ici  toutes les abréviations que vous souhaitez utiliser dans le document.
% les entrées seront triées

%\newacronym{1d}{1D}{une dimension}
%\newacronym{2d}{2D}{deux dimensions}
%\newacronym{gpu}{GPU}{{\it Graphics Processing Unit}, carte graphique}
%\newacronym{cpu}{CPU}{{\it Central Processing Unit}, unité centrale de traitement (UCT)}
\newacronym{cp}{CP}{{\it Phase Congruency}, congruence de phases}

% La commande \glsaddallunused qui est juste avant \end{document} dans le fichier .tex principal fait en sorte que toutes les entrées 
% d'abréviations définies dans le fichier prelim sont incluses dans la liste
% si ce comportement n'est pas souhaité, mettrze en commentaires.

% Si cette ligne est en commentaire, les abréviations seront listées dans la table des abréviations 
% si elles sont utilisées dans le texte seulement. Pour ce faire, vous devez utiliser la
% commande \gls(etiquette) à chaque fois dans le texte.
% en début de chapitre, utiliser \glsentryfull{edp} à la première utilisation.
% consulter glossariesbegin.pdf dans la documentation du paquetage glossaries 
% ou la page LaTeX/Glossary dans le wikibook sur Latex (https://en.wikibooks.org/wiki/LaTeX/Glossary)

% Précisions
% Don’t use \gls in chapter or section headings as it can have some unpleasant
% side-effects. Instead use \glsentrytext for regular entries and one of
% \glsentryshort, \glsentrylong or \glsentryfull for acronyms.
% Alternatively use glossaries-extra which provides special commands for use in
% section headings and captions, such as \glsfmtshort{⟨label⟩}.
