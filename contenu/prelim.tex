% -------------------------------------------------
% PAGE DE TITRE
% -------------------------------------------------

%
% ATTENTION : ne pas utiliser les macros \title \author ou \date !!!
%
\auteur{Vivien Gagliano}
\titre{Synthèse de texture procédurale pour environnement virtuel}

% n'existe que si la thèse est rédigée en anglais (pas obligatoire)
\englishTitle{Title in english}


% n'existent que si l'option cotutelle est présente
\institutionPartenaire{Université XYZ} 
\departementPartenaire{au département (centre, institut) ABC} 
\gradePartenaire{docteur en informatique} 

%\dedicace{À quelqu'un de significant} %%% ajouter un item de dédicace

% -------------------------------------------------

% DESCRIPTION DU SOMMAIRE (EN FRANCAIS) -----------
\sommaire{Écrire ici votre sommaire en français.}
\motsCles{Mettre vos mots-clés ici séparés par des virgules}

%  optionnel, valide si votre document est en anglais 
%  optionnal, valid only if your document is written in english, you have to provide a french abstract 
%  option	english du document
\abstract{Write here your abstract in english}
\keywords{Put your keywords here separated by commas}


% REMERCIEMENTS -----------------------------------
\remerciements{Ajoutez ici vos remerciements. Those can be in english if the document is in english.}

% LISTE DES ABREVIATIONS --------------------------
%% Insérez ici  toutes les abréviations que vous souhaitez utiliser dans le document.
% les entrées seront triées

\newacronym{etiquette}{ABR}{Expression à remplacer}
\newacronym{edp}{EDP}{Équations aux dérivees partielles}
\newacronym{sw}{SW}{Star Wars}
\newacronym{st}{ST}{Star Trek}
\newacronym{bv}{BV}{Babylon V}
\newacronym{1d}{1D}{une dimension}
\newacronym{2d}{2D}{deux dimensions}
\newacronym{gpu}{GPU}{{\it Graphics Processing Unit}, processeur graphique}
\newacronym{cpu}{CPU}{{\it Central Processing Unit}, processeur central}

% La commande \glsaddallunused qui est juste avant \end{document} dans le fichier .tex principal fait en sorte que toutes les entrées 
% d'abréviations définies dans le fichier prelim sont incluses dans la liste
% si ce comportement n'est pas souhaité, mettrze en commentaires.

% Si cette ligne est en commentaire, les abréviations seront listées dans la table des abréviations 
% si elles sont utilisées dans le texte seulement. Pour ce faire, vous devez utiliser la
% commande \gls(etiquette) à chaque fois dans le texte.
% en début de chapitre, utiliser \glsentryfull{edp} à la première utilisation.
% consulter glossariesbegin.pdf dans la documentation du paquetage glossaries 
% ou la page LaTeX/Glossary dans le wikibook sur Latex (https://en.wikibooks.org/wiki/LaTeX/Glossary)

% Précisions
% Don’t use \gls in chapter or section headings as it can have some unpleasant
% side-effects. Instead use \glsentrytext for regular entries and one of
% \glsentryshort, \glsentrylong or \glsentryfull for acronyms.
% Alternatively use glossaries-extra which provides special commands for use in
% section headings and captions, such as \glsfmtshort{⟨label⟩}.
