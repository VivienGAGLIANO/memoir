\chapter[Article soumis ou en préparation]
{Titre long en français d'un chapitre contenant un article soumis ou en préparation}
\label{chap:soumis}

\begin{authorsArticle} % dans l'ordre de la publication
	\begin{description}
		\item[\large nom 1] Affiliation
		\item[\large nom 2] Affiliation
		\item[\large ...] 
	\end{description}
\end{authorsArticle}
%\authorArticle{
	%}

\begin{abstractArticle}
	Résumé de l'article en français. Ce résumé peut être la traduction de l'article s'il est en anglais.
\end{abstractArticle}
% resumeArticle{}

\begin{contributions}
	Décrivez dans cette section les contributions de l'article.
\end{contributions}

\begin{commentairesArticle}
	Décrivez dans cette section la part de chacun.e des auteur.e.s
\end{commentairesArticle}

% - choisir la bonne option parmi les deux suivantes
\modeAnglais
%\modeFrancais
% --------------------------------------------

% IMPORTANT : ARTICLE TEL QUE SOUMIS
\titleArticle{Titre de l'article soumis}

\keywordsArticle{ keywords }

Intro de l'article ...

\section{Section Title }

Vous devez introduire la section.

\subsection{Sub-section Title}

Vous devez introduire la  sous-section.

\subsection{Sub-section Title}

Vous devez introduire la  sous-section.

\subsubsection{Sub-sub-section Title}

Vous devez introduire la  sous-sous-section.

\paragraph{A paragraph:} example

\subparagraph{A sub-paragraph:} exemple

\subsubsection{Sub-sub-section Title}

Vous devez introduire la  sous-sous-section.

\subsection{Sub-section Title}

Vous devez introduire la  sous-section.
