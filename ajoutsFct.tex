% -------------------------------------------------
% PACKAGES SUPPLEMENTAIRES au besoin
% -------------------------------------------------

% Les paquets déjà chargés sont les suivants :
%-->    \usepackage[utf8]{inputenc} ou \usepackage[latin1]{inputenc}
%-->    \usepackage[T1]{fontenc}
%-->    \usepackage{lmodern}
%-->    \usepackage{enumerate}
%-->    \usepackage{amsfonts, amsmath, amssymb, stmaryrd, latexsym}
%-->    \usepackage{xspace, setspace}
%-->    \usepackage{array}
%-->    \usepackage[dvipsnames,usenames]{color}
%-->    \usepackage[table]{xcolor}
%-->    \usepackage{wrapfig, epsfig}
%-->    \usepackage[english,frenchb]{babel}
%-->    \usepackage{frbib} % bibliography en francais
%-->    \usepackage{fancyhdr}
%-->    \usepackage{geometry}
%-->    \usepackage{ulem} % for sout and underline
%-->    \usepackage{times,helvet,courier}
%-->    \usepackage{listings} (pour le code source)
%-->    \usepackage[pdftex, colorlinks, linkcolor=blue]{hyperref}
%-->    \usepackage[square, numbers]{natbib}
%-->    \usepackage[nomain, automake, abbreviations, nonumberlist, toc]{glossaries-extra} % abréviations
%-->    \usepackage{booktabs} 
%-->    \usepackage{doi}
%-->    \usepackage{refstyle}   % permet de faciliter les références croisées et le changement de langue
%-->    \usepackage[final]{pdfpages} % permet l'inclusion de pdf dans le document (voir contenu/articlePublie.tex)
%-->    \usepackage[a-1b]{pdfx}	

%\usepackage{biblatex}

%\usepackage{splitbib}
%\begin{category}[A]{First category}
%	\SBentries{deschenes98,BuDo.99-guide-B}
%\end{category}
%\begin{category}[B]{Second category}
%		\SBentries{Abr.96-BBook,Hoa.85-CSP,Jac.83-JSD,Mil.89-CCS}
%\end{category}

%% =====================================
%% Ajoutez, enlevez ou adaptez ICI
%% =====================================
% \usepackage[nooneline]{subfigure} % deprecated
\usepackage{graphicx}
\usepackage{datetime}
\usepackage{algorithm}
\usepackage{algpseudocode} % permet de faire du pseudo code en francais
\usepackage{amsmath}
\usepackage{glossaries}
%\usepackage{caption}
\usepackage{subcaption} % subfigures
% \usepackage{soul} % strike through % workn't
\usepackage{float}
\usepackage{mathrsfs} % stylised H for Hilbert tranform
\usepackage{tikz} % draw in LaTex
\usepackage{pgfplots} % draw plots directly in LaTex
\pgfplotsset{width=6cm, height=4.5cm, compat=1.9}
\usepgfplotslibrary{external} % externalize plot compute for faster render time
\tikzexternalize

\if@english 
	% defaut du paquet
\else % vous pouvez adapter au besoin
	\algrenewcommand\algorithmicwhile{\textbf{tant\_que}}
	\algnewcommand\algorithmiccase{\textbf{}}
	\algnewcommand\algorithmicswitch{\textbf{selon}}
	\algnewcommand\algorithmicelsif{\textbf{sinon si}}
	\algrenewcommand\algorithmicdo{}%\textbf{faire}
	\algrenewcommand\algorithmicif{\textbf{si}}
	\algrenewcommand\algorithmicelse{\textbf{sinon}}
	\algrenewcommand\algorithmicthen{\textbf{alors}}
	\algrenewcommand\algorithmicend{\textbf{fin}}
	\algrenewcommand\algorithmicfunction{\textbf{fonction}}
	\algrenewcommand\algorithmicfor{\textbf{pour}}
	\algrenewcommand\algorithmicrepeat{\textbf{répéter}}
	\algrenewcommand\algorithmicuntil{\textbf{jusqu'à}}
	\algrenewcommand\algorithmicreturn{\textbf{retourner}}
	\algrenewtext{EndFunction}{\algorithmicend}
	\algrenewtext{EndFor}{\algorithmicend\_\algorithmicfor}
	\algrenewtext{EndCase}{\algorithmicend\_\algorithmiccase}
	\algrenewtext{EndIf}{\algorithmicend\_\algorithmicif}
	\algrenewtext{EndSwitch}{\algorithmicend\_\algorithmicswitch}
	\algrenewtext{EndWhile}{\algorithmicend\_\algorithmicwhile}
	\algdef{SE}[WHILE]{While}{EndWhile}[1]{\algorithmicwhile\ #1\ \algorithmicdo}{\algorithmicend\_\algorithmicwhile}%
	\algdef{SE}[SWITCH]{Switch}{EndSwitch}[1]{\algorithmicswitch\ #1}{\algorithmicend\_\algorithmicswitch}%
	\algdef{SE}[CASE]{Case}{EndCase}[1]{\algorithmiccase\ #1 :}{\algorithmicend\_\algorithmiccase}%
	\algtext*{EndCase}
	\algrenewcommand\algorithmicrequire{\textbf{Pré-condition}}
	\algrenewcommand\algorithmicensure{\textbf{Post-condition}}
        \DeclareMathOperator{\sgn}{sgn}
        \DeclareMathOperator{\hil}{\mathscr{H}}
\fi

% -------------------------------------------------
% COMMANDES SUPPLEMENTAIRES
% -------------------------------------------------

%---- pour l'anglais ----
\newcommand{\ie}{{i.e.}\xspace}
\newcommand{\eg}{{e.g.}\xspace}
\newcommand{\cf}{{\it cf.}\xspace}

%---- pour le francais ----
\newcommand{\cad}{{c'est-\`a-dire}\xspace}
\newcommand{\Cad}{{C'est-\`a-dire}\xspace}
\newcommand{\etc}{{\it etc.}\xspace}

%---- Math ----
\newcommand{\N}{\ensuremath{\mathbb{N}}\xspace}
\newcommand{\Z}{\ensuremath{\mathbb{Z}}\xspace}

% Nouvelles d\'efintion d'environnement de th\'eor\`eme et de d\'efinition.
\newtheorem{frtheoreme}{Th\'eor\`eme}[section]
\newtheorem{frlemme}[frtheoreme]{Lemme}
\newtheorem{frprop}[frtheoreme]{Proposition}
\newtheorem{frcoro}[frtheoreme]{Corollaire}
\newtheorem{frdefinition}{D\'efinition}[section]
